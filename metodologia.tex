
\chapter{Materiais e Métodos}
\label{metodologia}
Este capítulo descreve o modo como a pesquisa e o desenvolvimento do framework
 sugerido na \autoref{objetivos} será conduzida.

\section{Tecnologias utilizadas}
A escolhas das tecnologias seguem de acordo com a arquitetura do sistema.

\subsection{Javascript}
Para desenvolvimento do \emph{Framework} será utilizada a linguagem \emph{Javascript},
uma linguagem altamente difundida para desenvolvimento de páginas web.

Nos últimos anos a linguagem passou a ser usada também para desenvolvimento de
aplicações para servidor.

\subsection{Node.js}
Será utilizado o \emph{Node.js} como plataforma de execução do framework, ele
conta com recursos nativos do Sistema Operacional o que o torna extremamente
viável para o framework.

\subsection{Desenvolvimento}
Ao final do desenvolvimento, deveremos ter uma Aplicação, para realizar o
processamento das tarefas distribuídas, e um \emph{Framework} capaz de distribuir
tarefas em outras máquinas.


\subsubsection{Conexão}
O \emph{Framework} em sua inicialização receberá as informações, como
endere\c{c}o e detalhes de autenticação dos servidores que contém a aplicação que
processará as tarefas.

\subsubsection{Distribui\c{c}\~{a}o de tarefas}
O \emph{Framework} utilizará classes para:
\begin{itemize}
  \item Criação das tarefas;
  \item Configuração das tarefas;
  \item Distribuição das tarefas;
\end{itemize}
\subsubsection{Aplicação para execução das tarefas}
A aplicação para distribuição também será feita utilizando \emph{Javascript} e
\emph{Node.js}. Ela interpretará a tarefas recebida e responderá para o
\emph{Framework} se ela foi executada corretamente ou não. Em caso de
\emph{MapReduce} também responderá com os dados processados.
