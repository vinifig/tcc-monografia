\chapter{Cronograma}
\label{cronograma}
O cronograma do trabalho está divido em duas tabelas: \autoref{tcc-1} sobre as entregas na disciplina de TCCI
e a \autoref{tcc-2} as entregas na disciplina de TCCII.

\begin{table}[h]
  \begin{center}
  	\caption{\label{tcc-1}Cronograma de atividades para o TCCI}
  	\begin{tabular}{c{4cm} c{2cm} c{2cm} c{2cm} c{2cm} c{2cm}}
      \hline
      Atividade & Abr & Mai & Jun & Jul & Ago \\
      \hline
      \hline
      Levantamento bibliográfico inicial & X & X & X \\
      \hline
      Definição das tecnologias & X & X & X \\
      \hline
      Definição dos tipos de tarefas &  & X \\
      \hline
      Definição da Interface de Comunica\c{c}\~{a}o &  & X \\
      \hline
      Documentação da arquitetura do sistema distribuído & & X & X \\
      \hline
      Desenvolvimento do \emph{Framework} & & & X & X & X \\
      \hline
      Desenvolvimento da Aplicação & & & X & X & X \\
      \hline
      Desenvolvimento de modelo de tarefa para teste & & & & X  \\
      \hline
      Teste e documentação do que foi desenvolvido & & & &  & X \\
      \hline
  	\end{tabular}
  	\legend{Fonte: Elaborado pelo autor.}
  \end{center}
\end{table}

\begin{table}[h]
  \begin{center}
  	\caption{\label{tcc-2}Cronograma de atividades para o TCCII}
  	\begin{tabular}{c{5cm} c{2cm} c{2cm} c{2cm} c{2cm} c{2cm}}
      \hline
      Atividade & Set & Out & Nov & Dez & Jan \\
      \hline
      \hline
      Desenvolvimento dos modelos de tarefa restantes & X & X & X & X & X\\
      \hline
      Pesquisa de aplicações para utilização do \emph{Framework}  &  & X & X \\
      \hline
      Testar o \emph{Framework} nas aplicações escolhidas & & X & X & X \\
      \hline
      Teste finais e validação & & & & & X \\
      \hline
  	\end{tabular}
  	\legend{Fonte: Elaborado pelo autor.}
  \end{center}
\end{table}
