
\chapter{Fundamentação Teórica}
\label{fundamentacao-teorica}

% \subsection{Tipos de contadores}
% Não existe apenas um método para contar o número de pessoas. As principais
% diferenças entre os contadores estão em: área de cobertura, volume e tecnologia
% utilizada. Segundo \citeonline{wikipedia2017} e \citeonline{Ipsos} os principais métodos de
% contagem são:
%
% \begin{itemize}
%   \item \textbf{Feixes infravermelhos:} são colocados
% na entrada de lojas emitindo um feixe infravermelho entre os seus extremos,
% quando alguém interrompe o feixe, uma entrada é contada. A área de cobertura é
% pequena e o volume de pessoas que ele permite passando pela porta ao mesmo
% tempo é baixíssima;
%   \item \textbf{Câmeras termais:} o uso de sensores térmicos e
% processamento de imagens. Normalmente,
% são posicionados no teto para que a imagem capture a temperatura das pessoas
% e compare com a do ambiente. Este sistema permite alto volume de tráfego e instalação em entradas complexas.
%   \item \textbf{Vídeo:} Utilização de algoritmos complexos, inteligência artificial
%    e o processamento de imagens (2D e 3D). A área de cobertura
%   pode ser medida de acordo com o uso de câmeras e o volume permitido varia de acordo com os algoritmos.
%   \item \textbf{Wi-Fi:} utiliza o receptor Wi-Fi para pegar \emph{frames} únicos de gerenciamento Wi-Fi emitidos por dispositivos
%   dentro do alcance. Ideal para áreas onde o volume de pessoas é esparso ou incerto.
% \end{itemize}

% As escolha de um contador varia de acordo com a complexidade da entrada do lugar, períodos de captura do tráfego de pessoas,
% volume de pessoas por período, área de cobertura, precisão desejada, preço, entre outros \cite{trafsys} \cite{Axper2017}.
%
% \subsection{Área acadêmica}
% A principal técnica de contagem de pessoas pesquisada é por câmeras e processamento de imagens. Entretanto,
% as pesquisas diferenciam-se por técnicas de computação utilizadas. Alguns exemplos são:
%
% \begin{itemize}
%
%   \item \textbf{Robusto e leve:} com o objetivo de fornecer segurança para ambientes internos
%   o trabalho de \citeonline{Kim2002} preza por um sistema que seja robusto suficiente para garantir as metas, mas
%   não seja tão pesado do ponto de vista de algoritmos e demanda de hardware. O sistema reconhece o movimento de pessoas
%   ao longo de várias direções através de uma única câmera e um processador Pentium IV, assim ele estima e rastreia uma "caixa"  ao redor de cada indivíduo
%   para identificá-lo na imagem;
%
%   \item \textbf{Melhora no processamento de imagens e ruídos:} as pesquisas de \citeonline{Luo2016} e \citeonline{Hou2011} consideram
%   a queda de desempenho de sistemas de contagem em ambientes com multidões, oclusões (sombreamento/luminosidade
%   em cada quadro do vídeo) e informações de fundo complexas. O primeiro artigo propõe uma abordagem de cenas \emph{indoor}
%   que leva em conta multidões estacionárias (paradas) ou em movimento. O sistema detecta a multidão e separa
%   os ruídos. Depois, estima-se o número de pessoas através de "ombro-cabeça". Por fim, para reduzir as oclusões,
%   há um filtro que separa quadro por quadro do vídeo e faz um tratamento. Já o segundo, foca em subtrair o fundo, estima
%   o número de pessoas e utiliza técnicas para identificar as pessoas em imagens de baixa resolução;
%
%   \item \textbf{Múltiplos recursos:} os artigos de \citeonline{Venkatesh2015} e \citeonline{Ma2012} consideram múltiplos recursos para contar pessoas
%   em ambientes densos. O primeiro utiliza, principalmente, técnicas matemáticas e técnicas de filtros e imagens para estimar. Já o segundo, utiliza
%   múltiplas câmeras e vários níveis de textura para lidar com aparência humana e posições.
%
% \end{itemize}
%
%   As principais caraterísticas de sistemas de contagem que os artigos levantados focaram e presaram foram: movimentação das pessoas,
%   ambientes de multidão e processamento em tempo real.
%
% \subsection{Produtos na área empresarial}
% \label{produtos-empresas}
% Esta seção apresenta alguns contadores de algumas empresas. Na \autoref{density}, a empresa Density oferece a contagem a partir de um
% dispositivo localizado no topo da entrada que processa imagens 2D \cite{Density2017}. Já na \autoref{axper}, a empresa Axper além de oferecer
% o processamento de imagens 2D como a Density, oferece também um dispositivo que processa imagens 3D, cobrindo todo o ambiente \cite{Axper2017}.
%
% \begin{figure}[htb]
%   \caption{\label{density}Processamento de imagens 2D - Density People Counter}
%   \begin{center}
%     \includegraphics[width=0.40\textwidth]{img/density.png}
%   \end{center}
%   \legend{Fonte: \citeonline{Density2017}.}
% \end{figure}
