
\chapter{Introdução}
\label{introducao}

Em um ambiente em que a demanda computacional para processamento de dados e
execução de tarefas ultrapassa a capacidade computacional é necessário encontrar
maneiras diferentes de atender a demanda, como a utilização de sistemas
distribuídos, seguindo o conceito de divisão e conquista para atender as
necessidades do ambiente.

\section{Problema}
Com o crescimento de linguagens de interpretação é
importante que as mesmas tenham maneiras práticas de processar sua demanda
computacional, o que se torna uma tarefa difícil quando não se encontram pacotes
e bibliotecas de processamento distribuído dentre os grandes frameworks destas
linguagens.

\subsection{Solução proposta e motivação}
Tendo em mente o problema previamente descrito, o foco é o desenvolvimento
de um \emph{Framework} que permita o processamento distribuído para \emph{Node.js}, uma
plataforma de aplicações server-side com \emph{Javascript}, com recursos para
passar para outras máquinas tarefas e sua forma de execução.

\section{Objetivos}
\label{objetivos}

\subsection{Objetivos Gerais}
Este trabalho tem como objetivo desenvolver um \emph{Framework} para \emph{Node.js} que
permita o seu utilizador realizar o processamento distribuído de suas tarefas.

\subsection{Objetivos específicos}
\begin{itemize}
  \item Estudar as ferramentas que realizam processamento distribuído.
  \item Definir os tipos de tarefas a serem distribuídas.
  \item Definir interfaces para descrever as tarefas a serem distribuídas.
  \item Desenvolver um \emph{Framework} para distribuir as tarefas.
  \item Desenvolver uma aplicação para executar as tarefas distribuídas
  \item Testar a utilização da biblioteca em aplicações \emph{Javascript} de código aberto
  e de alto custo computacional
\end{itemize}

\section{Organização do trabalho}
O presente trabalho divide-se em seções, sendo esta a primeira (Introdução) e os demais na seguinte ordem:

\begin{itemize}
  \item \textbf{Fundamentação Teórica:} apresentação do embasamento teórico
  envolvidos no trabalho, e soluções semelhantes;
  \item \textbf{Metodologia:} ferramentas escolhidas para o desenvolvimento,
  e módulos do sistema;
  \item \textbf{Cronograma:} tarefas que serão realizadas dentro de seus
  respectivos prazos.
\end{itemize}
